\section*{Introduction}
In this work we develop and test a parallel algorithm to count the number of solutions for the $n$-queens problem.
This problem is about setting $n$ queens on a $n\times n$ chess-board, such that no queen is in the same row, column or diagonal as any other queen.
Such solutions can be found with a tree search.
The nodes of this tree are configurations of the board the root node is the empty board.
The children of a node always contain the same configuration of the board as the node itself, but with a new queen in the first empty row.
In order to decrease the branching factor, sub-trees are pruned off as soon as one queen on the board violates any of the given constraints, i.e., is in the same row, column or diagonal as another queen.

Section 1 shows the algorithms that were used.
We first talk about the sequential approach, before we go into detail on how to construct the parallel algorithm.
Section 2 introduces the experiments with which we tested the algorithms.
The results are shown in Section 4, and discussed in Section 5.
